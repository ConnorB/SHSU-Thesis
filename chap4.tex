%-----------------------------------------------------------------------
% Beginning of chap4.tex
%-----------------------------------------------------------------------
%
%%%%%%%%%%%%%%%%%%%%%%%%%%%%%%%%%%%%%%%%%%%%%%%%%%%%%%%%%%%%%%%%%%%%%%%%

\chapter[DISCUSSION]{Discussion}
\vspace{-12pt}

% We will want to start with a paragraph re-capping your findings and then go into discussing those main findings.
 Across the coupled land use and precipitation regime gradients, all streams were heterotrophic with rates of ER exceeding GPP. Sites had very low GPP (0.12 to 0.78 \unit{\goxy}) compared to ER across the coupled gradients, with no strong apparent seasonal trends and more daily and year to year variation. There was no apparent trend in nutrient concentrations or GPP. However, ER and DOC generally increased as agricultural land increased and non-agricultural vegetation decreased, moving from semi-arid to mesic.
 
\section{Ecosystem Metabolism}
%see some of the work out of Jen Tanks lab -- she had 2 students work on metabolism in ag streams in Indiana.  We have ag watersheds. They found high rates of GPP (and ER) --- Roley is one and Natalie Griffiths is another

%Found one paper from Griffiths, Roley, Tank, Whiles, Rosi, and Beaulieu for 6 midwestern row-crop drained streams  



%site figures and tables as needed to reference the data you are discussing

% don't site AJU 2018 29 GPP, find different site, 29 may not be real (prob half that, bad K), currently citing stream 6 1st order stream as 25 gO2m-2d-1 GPP
% ask AJU about making Fig 1 from AJU 2018, would be useful in thesis appendix, or in MS

Texas coastal plain streams appear to have relatively low GPP compared to other stream ecosystems (Figure~\ref{fig:GPPBox}). For instance, mean GPP in tropical streams with an open canopy was 2.09 \unit{\goxy} and in a small forested head water stream in Tennessee mean GPP was 1.4 \unit{\goxy} peaking at 10.80 \unit{\goxy} \cite{roberts_multiple_2007, marzolfEcosystemMetabolismTropical2021}. Sub-alpine streams appear to have greater GPP compared to the Texas coastal plain streams as well where GPP from 12 Austrian sub-alpine streams peaked at 25 \unit{\goxy} \cite{ulseth_climate-induced_2018}. Agricultural streams elsewhere also have greater GPP, where in six mid-western row-crop draining streams with high light availability and nutrients, mean GPP was 4.6 \unit{\goxy} peaking at 22.0 \unit{\goxy} \cite{griffithsAgriculturalLandUse2013}. In comparison, 33 Austrian streams with a mix of forested, agriculture, and urban land uses were slightly similar to slightly higher than Texas coastal plain streams where GPP ranged from 0.01 to 3.3 \unit{\goxy} \cite{fus_land_2017}. Rates of GPP from these Texas coastal streams were most similar to those of closed canopy tropical streams (0.57 \unit{\goxy}),  \cite{marzolfEcosystemMetabolismTropical2021}. Canopy cover could explain some of the limitation of light, for sites where canopy cover was low, turbidity likely decreased light reaching the benthos, where most of the primary production would take place (average turbidity ranged from 15-141 NTU). This suggest these streams were likely light limited, resulting in low rates of GPP \cite{hall_turbidity_2015,honiousTurbidityStructuresControls2021, marzolfEcosystemMetabolismTropical2021}. Another contributing factor to low GPP in these ecosystems may be caused by sandy substrate and frequent bed movement. Across the coupled gradients, a majority of sites had sandy substrate. In a desert stream in Arizona, low GPP (0.15 to 0.29 \unit{\goxy}) was attributed to bed movement caused by sandy substrate disrupting primary producers \cite{uehlingerHeterotrophicDesertStream2002}. Again in a rural Australian stream, low GPP (0 to 0.5 \unit{\goxy}) was also attributed to bed movement caused by sandy substrate \cite{atkinsonSedimentInstabilityAffects2008}. In streams with high bed movement, GPP is often suppressed due to disturbance to primary producers \cite{uehlingerEcosystemMetabolismDisturbance1998, uehlingerHeterotrophicDesertStream2002, atkinsonSedimentInstabilityAffects2008}. 

Within sites, daily variation of GPP exceeded that of seasonal variation (Figure~\ref{Fig:MetabStacked}). There was not strong seasonal variation or a peak of GPP in the spring or summer within sites. The lack of seasonal trends in GPP appears to be uncommon in other stream ecosystems. For instance, in a small forested Tennessee stream GPP peaked during spring before leaf out (0.01 to 10.80 \unit{\goxy}), which was attributed to an increase in light due to longer days in spring prior to canopy leaf out \cite{roberts_multiple_2007}. In 12 Austrian sub-alpine streams GPP peaked in spring after snow melt (0.01 to 25 \unit{\goxy}) which was also attributed to increased light \cite{ulseth_climate-induced_2018}. In 6 mid-western row-crop drained streams, GPP also peaked in the spring (0.1 to 22.0 \unit{\goxy}) which was also attributed to high light availability \cite{griffithsAgriculturalLandUse2013}. Again, in an analysis of 222 stream and rivers across the Unites States, seasonal changes in light and flow regimes were found to be the strongest drivers of GPP \cite{bernhardtLightFlowRegimes2022}.  In the subtropical Texas coastal plains region, where there is a more subtle shift in seasons, compared to temperate ecosystems, along with high turbidity in these streams, may be attributed to low GPP and lack of strong seasonal peaks in GPP seen in other stream ecosystems.

%Comment from AJU, think about substrate, these are sandy streams compared to more rock of pebble substrate (lots of bed movement, disrupting producers?). Think about Urs uhlinger scoured paper, hard substrate, maybe nancy grimm 1980s(?) paper, but used chambers

%Our ER was close to ER from the Marzolf tropical ag impacted stream, not overall review. Also found a negative relationship between SPR and ER, ag stream had 0.25\unit{\mg\per\l} SPR (highest in marzolf study) (similar to our P)

%we need a topic sentence here regarding ER -- something perhaps along the lines that ER rates from the TX coastal plain streams were slightly lower, but within range (?) of estimates of ER from other stream ecosystems.  Then go into comparing rates from other studies has you have done ---
ER rates from these Texas coastal plain streams were lower, but within range (-0.4 to -29.0 \unit{\goxy}) of estimates of ER from other stream ecosystems \cite{bernot_inter-regional_2010}. The median ER rates across these Texas coastal plain streams were similar to estimates of ER from an agriculturally impacted tropical stream in Costa Rica (-0.5 to -0.8 \unit{\goxy}) \cite{ortega-pieckAgriculturalInfluencesMagnitude2017}.
However, median ER of all sites was lower than estimates of ER from tropical streams with a mix of open and closed canopy (-4.30 \unit{\goxy}), six mid-western row-crop drained streams (-0.9 to -34.8 \unit{\goxy}), eight streams with a mix of agriculture, urban, and reference land uses across the Unites States (-2 to -8 \unit{\goxy}), a small forested head water stream in Tennessee (-0.99 to -16.01 \unit{\goxy}), and in 12 streams across an Austrian sub-alpine catchment (-0.04 to -54.2 \unit{\goxy}) \cite{marzolfEcosystemMetabolismTropical2021, bernot_inter-regional_2010, mulholland_inter-biome_2001, ulseth_climate-induced_2018, roberts_multiple_2007, griffithsAgriculturalLandUse2013}. As nutrients and DOC concentrations were relatively high to fuel microbial metabolism (Table \ref{tab: Nutrients}), I would have expected greater rates of ER; however, rates of ER were on the lower end of the range reported from other ecosystems \cite{bernot_inter-regional_2010, mulholland_inter-biome_2001, ulseth_climate-induced_2018, roberts_multiple_2007, griffithsAgriculturalLandUse2013}. These lower than expected rates of ER may be explained by the lack of hyporehic exchange caused by low channel slope. Streams with low slope have decreased hyporehic exchange where an estimated 50-85\% of of ER fluxes originate  \cite{stanfordEcosystemPerspectiveAlluvial1993, naegeliContributionHyporheicZone1997,fellowsWholeStreamMetabolism2001, mulhollandEvidenceThatHyporheic1997}.
 

% one reason why the SEM model does not work is because we have very little variation in GPP in ER at the monthly level and across sites -- so if nothing varies, the changes in our regional and proximal drivers wouldn't correlate with GPP and ER.

\section{DOC and Nutrients}

Across most sites, DOC concentrations increased along the precipitation gradient. This trend was expected, as increasing DOC concentrations are linked to increases of terrestrial primary production \cite{meybeckCarbonNitrogenPhosphorus1982, mulhollandGroupReportWhat1989, jonesLongtermPerspectiveDissolved1996}. However, the driest site, TRC and the second most mesic site, EMC did not follow this pattern. TRC is dominated by a dense, intact riparian zone comprised of native shrub vegetation, anoxic stream water (DO < 0.5 \unit{\mg\per\l}) 75\% of the time, and an average discharge of 0.03 m$^3$/s. With possible low hyporheic exchange, lack of mineralization of DOC may have caused the DOC concentrations to be higher than found at other sites. The second site that did not follow the pattern was EMC, this site has a watershed that is dominated by agriculture and has a sandy riparian zone, this may have led to a decrease in DOC compared to other sites. The sandy sediment in the riparian zone contains less organic carbon than riparian zones composed of soil \cite{boix-fayosSedimentFlowPaths2015}. The increase in DOC across the precipitation gradient did not drive an increase in ER. A temporal miss-match in the monthly sampling schedule potentially caused the inability to link interactions between DOC and ER.


%here we also would need a topic sentence to better encapsulate your discussion.  Rather than have SFC as your topic sentence -- 'Nutrient concentrations did not appear to drive variation in GPP and ER'.  As you are using this paragraph to 1. introduce discussion regarding nutrient concentrations and 2. the effect of said nutrients on rates of GPP and ER
Nutrient concentrations did not drive variation in GPP and ER. All sites, except SFC, had similar concentrations of ammonium amd nitrate, with  no apparent trend across the gradients (Table \ref{tab: Nutrients}). Sites receiving WWTP effluent had increased level of nitrate. SFC is predominantly waste water dependent, likely contributing to increased nitrate and phosphate concentrations. This increase in nutrients may have lead to SFC having the second highest ER for these coastal streams (-2.60 \unit{\goxy}). Generally, increases in nutrient concentrations fuel microbial respiration and lead to increased ER \cite{ramirezEffectsStreamPhosphorus2003}. However, the results from the SEM (Figure~\ref{Fig:SEMmetab}) indicate there was not a significant impact of nutrient concentrations on GPP or ER. Previous studies on eight and nine streams with a mix of agriculture, urban, and reference land uses across the Unites States and Puerto Rico and 83 streams across the global tropics have also been unable to relate rates of ecosystem metabolism to nutrient concentrations \cite{mulholland_inter-biome_2001,bernot_inter-regional_2010, marzolfEcosystemMetabolismTropical2021}. The findings from this study and others suggest weekly or monthly nutrient measurements may be insufficient for linking nutrient concentrations to changes in ecosystem metabolism and more frequent measurements of nutrient concentrations are needed to link the effects of nutrients to daily estimates of GPP and ER \cite{bernot_inter-regional_2010, fus_land_2017}.


\section{Discharge}

Across the precipitation and land use gradients, WMC had the highest discharge, while SFC had the most flow variability (Table~\ref{tab:Q}). All sites exhibited quick increases in discharge from frequent precipitation events, with equally quick decreases after precipitation events. Frequent precipitation events, which increase stream discharge, leads to scouring of primary producers and increases in turbidity, which may reduce GPP \cite{uehlinger_resistance_2000, hall_turbidity_2015}. However, the low GPP in the Texas coastal streams may already be light limited without increased turbidity from high flows. 


\section{Precipitation and Land use Gradients}

With monthly measurements of nutrients, DOC, and turbidity, there was a strong influence of land use proximal drivers of ecosystem metabolism, with land use driving changes in nitrate, phosphate, ammonium, and turbidity. However, there was not an influence of DOC, nitrate, phosphorous, ammonium, and turbidity on GPP and ER.  The inability to find a link between proximal drivers and ecosystem metabolism suggest the snap-shot sampling of nutrients, DOC, and turbidity was too limited to detect responses of ecosystem metabolism to changes in proximal drivers \cite{bernot_inter-regional_2010}. The low variation in GPP and ER (Table \ref{metab}) may have also prevented linking drivers to ecosystem metabolism. Additionally, these streams had intact riparian zones and were likely light limited, which may have limited the response of ecosystem metabolism to changes in land use \cite{jankowskiLandUseChange2021, bernhardtLightFlowRegimes2022, griffithsAgriculturalLandUse2013}.

\section{Conclusion}

%Did a quick lit search, tons of ppl refer to this area as subtropical with no citations

%I broke this into 2 paragraphs, because the flow felt off reading it all together. I may need help word smithing. But let me know what you think
While I was able to parse out the effect of regional drivers (land use and precipitation regimes) on proximal drivers of ecosystem metabolism (ie. nutrients, DOC, and turbidity), I was unable to elucidate the combined effects of precipitation and land use gradients on stream ecosystem metabolism. This suggest the two coupled gradients, land use and precipitation regime, may mask the combined effects on stream ecosystem metabolism. Future work should focus on the use of high frequency measurements of proximal drivers to attempt to parse out drivers of ecosystem metabolism in subtropical streams. 




A majority of research on stream ecosystem metabolism comes from temperate regions, with tropical and subtropical regions being understudied \cite{marzolfEcosystemMetabolismTropical2021}. These subtropical Texas coastal plains streams provide valuable insight into how subtropical streams may differ from their more studied temperate counterparts. These low estimates of GPP suggest that Texas coastal streams likely rely heavily on allochthonous rather than autochthonous carbon sources as a basal resource. These are slow, flat streams, which have low GPP and low to moderate ER. These streams do appear to function differently than their more temperate counterpart. But in light of global change, teasing apart what drives ecosystem metabolism will be of importance given that GPP and ER modulate carbon, nutrient fluxes, and even basal food resources.




% Also, we want to weave in the strengths and weaknesses of this study.  Often it is fair to discuss the potential weaknesses of the study towards the end -- Also think about future directions as well. I like where your conclusion is going, that is the idea there -- now you may also want to think about what we would need to know in terms of discerning land use, for instance

%Also, as I am typing this -- I wonder if land use is similar to biomes. Referencing Bernhardt et al. Metabolic Regime paper and some of Walter Dodd's work - 'do streams have biomes', perhaps another reason we did not see a strong connection with ppt gradient and land use is because the phenology of the terrestrial ecosystem is not typed with essentially the phenology of the stream ecosystem. And as you allude in your conclusion text -- local drivers (light, riparian vegetation) may be of more importance than large-scale drivers.  I would have to read through Dodds paper again - but I think they concluded that streams did not have biomes. As for the metabolic regime of coastal TX streams -- dampened seasonal trends?  Just a few thoughts...



% With continued climate change, it is uncertain how stream ecosystem functions such as carbon storage and transport will respond.





\endinput

%-----------------------------------------------------------------------
% End of chap4.tex
%-----------------------------------------------------------------------
