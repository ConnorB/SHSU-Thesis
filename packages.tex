%%%%%%%%%%
%%%%%%%%%%
% PACKAGES
%%%%%%%%%%
%%%%%%%%%%

%%%%%%%%%%%%%%%%%%%%%%%%%%%%%%%%%%%%%%%
% I have added the packages I need to provide the correct formatting. 
% I have also added some packages that are commonly used.
% If you need to add more packages then you should do it at the end of this list.
%%%%%%%%%%%%%%%%%%%%%%%%%%%%%%%%%%%%%%%
%
% This package defines commands to access bold math symbols.  through \bm
%\usepackage{bm} 


%
% This package provide some additional commands to enhance
% the quality of tables in LaTeX.
\usepackage{booktabs}
%
% This package defines a single environment tabu to make all kinds of tabulars in text or in math mode provided that they do not split across pages.
% tabu is more flexible that tabular, tabular*, tabularx and array and extends the possibilities.
\usepackage{tabu}
%
% delarray introduces delimeters \begin{array}({cc})
% The tabulary environment expands specific columns to meet the table’s width requirement 
\usepackage{array,delarray,tabulary,colortbl}
%
% enumitem package permits adjustment of list parameters
%\begin{enumerate}[topsep=0pt, partopsep=0pt]
%to suppress all spacing above and below your list
%\begin{enumerate}[itemsep=2pt,parsep=2pt]
%to set spacing between items and between paragraphs within items
%\begin{enumerate}[label=(\emph{\alph*})] or \Alph, \arabic, \roman and \Roman
\usepackage{enumitem} 
%
% Provides:
% the IEEE itemize, enumerate and description list environments
% the complete IEEEeqnarray family for producing multiline equations
% as well as matrices and tables, including the IEEEeqnarray support
% commands.
% This package clashes with enumitem
% \usepackage[retainorgcmds]{IEEEtrantools}
%
% Create tabular cells spanning multiple rows.
\usepackage{multirow}
%
% The package rotating gives you the possibility to rotate any object of an arbitrary angle. 
\usepackage{rotating}
%
% Provides command adjustwidth to locally change the margins of the text, and commands \changetext and \changepage for more radical changes to the page design mid-way through a document.
\usepackage{changepage}
%
% This package provides a verbatim command which is useful in case you want to 
% include the outcome of a computation exactly like it is displayed on the computer.
\usepackage{verbatim}
%
% float provides the option [H] for floating environments
% \afterpage, that causes the commands specified in its argument to be expanded
% after the curent page is output 
% afterpage places the float 'somewhere close' 
% \afterpage{\clearpage\begin{figure}[H] ...\end{figure}}
%\usepackage{afterpage,float}
%
% Using the package listings you can add non-formatted text as you would do with \begin{verbatim} but its main aim is to include the source code of any programming language within your document. 
%\usepackage{listings}
%
% \xspace should be used at the end of a macro designed to be used mainly in text. 
% It adds a space unless the macro is followed by certain punctuation characters.
%\usepackage{xspace}
%
%to anable \mathscr command for math script fonts
\usepackage[mathscr]{euscript}
%
% The mathrsfs package uses a really fancy script font (the name stands for "Ralph Smith's Formal Script") which is already part of most modern TeX distributions. 
% The package creates a new command \mathscr.
%\usepackage{mathrsfs}
%
% Martin Vogel's Symbols (marvosym) font.
% \usepackage{marvosym}
%
% Fonts and macros for IPA phonetics characters.
%\usepackage{tipa}
\usepackage{makecell}
%\usepackage{enumitem}

\usepackage{lscape}
\usepackage[section]{placeins}

%Nice SI units
\usepackage{siunitx}


\endinput

%-----------------------------------------------------------------------
% End of packages.tex
%-----------------------------------------------------------------------
