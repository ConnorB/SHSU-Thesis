%-----------------------------------------------------------------------------
% Beginning of vita.tex
%-----------------------------------------------------------------------------
%
%%%%%%%%%%%%%%%%%%%%%%%%%%%%%%%%%%%%%%%%%%%%%%%%%%%%%%%%%%%%%%%%%%%%%%%%
\clearpage
\phantomsection
\setlength{\parindent}{0pt}
\addcontentsline{toc}{chapter}{VITA}
\begin{singlespace}
\setstretch{1.25}
\begin{center}{\textbf{VITA}}\end{center}
\par
\par
% Put you vita here (below is a sample.)
\begin{center}Connor L. Brown\end{center}

\textbf{EDUCATION}\\
\smallskip\textbf{M.S., Biology} 2019-Present
Sam Houston State University, Huntsville, Texas
Thesis:\\Ecosystem  metabolism  of  coastal  Texas  streams  across  precipitation regimes and land use gradients.\\
\textbf{Relevant Coursework:} Ichthyology, Invertebrate Zoology, Biogeography, Biostatistics, Hydrology, Stream Ecology

\medskip\textbf{B.S., Natural Resources Management} 2016-2018
Texas Tech University, Lubbock, Texas\\
\textbf{Concentration:} Aquatic and Fisheries Biology\\
\textbf{Minor:} Geographic Information Science and Technology\\
\textbf{Relevant Coursework:} Freshwater Bioassessment, Watershed Planning, Fisheries Conservation and Management, Introduction to Geographic Information Systems, Advanced Geographic Information Systems, Cartographic Design, Spatial Analysis, GPS Mapping, Aerial Photo Interpretation, Integrated Natural Resources Management Skills, Wildlife and Vegetation Techniques, Diversity of Life, Genetics, Introduction to Conservation Science.

\bigskip\textbf{PUBLICATIONS}

\smallskip\textbf{Brown, C. L.} Delaune, K. D., Pease, A. A. 2022. \textit{Benthic Macroinvertebrate Assemblage Structure in Salinized Reaches of the Lower Pecos River, Texas.} (In Review.). The Southwestern Naturalist
\bigskip

\textbf{EXPERIENCE}

\smallskip\textbf{2020-2021} Graduate Research Assistant\\Stream and Biogeochemistry Lab, Department of Biological Sciences, Sam Houston State University.\\Managed data collection, analyzed time series data, estimated ecosystem metabolism using R, and collected macroinvertebrates and fish for the NSF Funded project Thresholds in Ecosystem Responses to Rainfall Gradients (TERRG).

\medskip\textbf{2019} Graduate Research Assistant\\Texas Research Institute for Environmental Studies, Sam Houston State University\\Worked on a funded grant from the Texas Military Department to collect invertebrate and habitat data, identified terrestrial and aquatic invertebrates, and pinned specimens.

\medskip\textbf{2019-2020} Teaching Assistant\\Department of Biological Sciences, Sam Houston State University\\Taught three sections of Ecology Lab and Environmental Science. This included giving a lecture over the material, leading lab activities, and assisting in students' knowledge of ecology and environmental science.

\medskip\textbf{2016-2018} Undergraduate Researcher\\Department of Natural Resources Management, Texas Tech University\\Proposed and conducted a research project on the Pecos River. Field work included setting up Hester-Dendy samplers at three sites along the Pecos and taking habitat measurements. Lab work included sorting macroinvertebrates and identifying them using dichotomous keys.

\medskip\textbf{2016} Undergraduate Internship\\Department of Natural Resources Management, Texas Tech University\\Worked closely with a graduate student on their dissertation with both field and lab work. Field work included setting up transects, collecting macroinvertebrates, fish, water sampling, as well as taking habitat measurements. Lab work included using dichotomous keys to identify macroinvertebrates as well as extracting environmental DNA from water samples.

\bigskip\textbf{PRESENTATIONS}

\smallskip\textbf{Brown, C. L.}, Carvallo, F., Frazier, C., Groff, C.M., Jenkins, V., Kinard, S. K., Solis, A.T., Strickland, B., Hogan, J. D., Patrick, C. J., Whiles, M. R., Zanden, H. V., Ulseth, A. J., 2021. Ecosystem metabolism of coastal Texas streams across precipitation and land use gradients. Society for Freshwater Science.

\medskip\textbf{Brown, C. L.}, Delaune, K. D., Pease, A. A., 2018. Spatial and Temporal Variation in Benthic Macroinvertebrate Assemblages Structure in Salinized Reaches of the Pecos River. Desert Fishes Council, Death Valley, California.

\medskip\textbf{Brown, C. L.}, Delaune, K. D., Pease, A. A., 2018. Macroinvertebrate community structure on the lower Pecos River. Southwestern Association of Naturalists, San Marcos, Texas.
\end{singlespace}
\endinput
%-----------------------------------------------------------------------------
% End of vita.tex
%-----------------------------------------------------------------------------
