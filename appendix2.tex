%-----------------------------------------------------------------------------
% Beginning of appendix2.tex
%-----------------------------------------------------------------------------
%
%%%%%%%%%%%%%%%%%%%%%%%%%%%%%%%%%%%%%%%%%%%%%%%%%%%%%%%%%%%%%%%%%%%%%%%%

\chapter{APPENDIX B}
\label{cha:app2}

% Appendices MUST NOT have sections or subsections
%\vspace{-12pt}
%\section{Some Definitions}

\indent \textbf{Anomaly detection.} Detecting anomaly for given statistical models.

\textbf{Classification of objects (Spectral classification).} Classifying
objects in a HSI data set.

\textbf{Demixing.} Finding material components in a raster cell.

\textbf{Electromagnetic radiation (EMR).} The energy in the form of
electromagnetic waves.

\textbf{Electromagnetic spectrum.} The entire family of electromagnetic
radiation, together with all its various wavelengths.

\textbf{Endmember spectra.} The \textquotedblleft pure\textquotedblright%
\ spectra that contribute to mixed spectra.

\textbf{Fused images.} A fused image is a combination of the HSI image and the
HRI image (to be mentioned below). It is usually the best because of high
resolution from the HRI camera and color information from the HSI sensor. This
combination results in sufficiently high image resolution and contrast to
facilitate image evaluation by the human eyes.

\textbf{High Resolution Imagery (HRI).} A high-resolution image (HRI) camera,
which captures black-and-white or panchromatic images, is usually integrated
in an HSI system to capture the same reflected light. However, the HRI camera
does not have a diffraction grating to disperse the incoming reflected light.
Instead, the incoming light is directed to a wider CCD (Charge-Couple Device) to capture more image
data. The HSI resolution is typically one meter per pixel, and the HRI
resolution is much finer: typically a few inches square per pixel.

\textbf{Hyperspectral imaging.} The imagery consists of a larger number of
spectral bands so that the totality of these bands is numerically sufficient
to represent a (continuous) spectral curve for each raster cell.

\textbf{Illumination factors.} The incoming solar energy varies greatly in
wavelengths, peaking in the range of visible light. To convert spectral
radiance to spectral reflectance, the illumination factors must be accounted.
Illumination factors taken into account includes both the illumination
geometry (angles of incoming light, etc.,) and shadowing. Other factors, such
as atmospheric and sensor effects, are also taken into consideration.

\textbf{Macroscopic and intimate mixtures.} Microscopic mixture is a linear
combination of its endmembers, while an intimate mixture is a nonlinear
mixture of its endmembers.

\textbf{Mixed spectra.} Mixed spectra, also known as composite spectra, are
contributed by more than one material components.

\textbf{Multi-spectral imaging.} The imaging bins the spectrum into a handful
of bands.

\textbf{Raster cell.} A pixel in a hyperspectral image.

\textbf{Reflectance conversion.} Radiance values must be converted to
reflectance values before comparing image spectra with reference reflectance
spectra. This is called atmospheric correction. The method for converting the
radiance to reflectance is also called reflectance conversion. The image-based
correction methods include \emph{Flat Field Conversion} and \emph{Internal
Average Relative Reflectance Conversion}. They apply the model $R_{1}=mR_{2},$
where $R_{1}$ is the reflectance, $R_{2}$ is the radiance, and $m$ is the
conversion slope. Some conversions also apply the linear model $R_{1}%
=-c+mR_{2},$ where $c$ is an offset that needs to be abstracted from the
radiance. The popular conversions are

\begin{enumerate}
\item \emph{Flat Field Conversion}. A flat field has a relatively flat
spectral reflectance curve. The mean spectrum of such an area would be
dominated by the combined effects of solar irradiance and atmospheric
scattering and absorption. The scene is converted to "relative" reflectance by
dividing each image spectrum by the flat field mean spectrum.

\item \emph{Internal Average Relative Reflectance (IARR) Conversion}. This
technique is used when no knowledge of the surface materials is available. The
technique calculates a relative reflectance by dividing each spectrum (pixel)
by the scene average spectrum.
\end{enumerate}

\textbf{Region segmentation.} Partitioning the spatial region of a
hyperspectral image into multiple regions (sets of pixels). The goal of
segmentation is to simplify and/or change the representation of an HSI image
into something that is more meaningful and easier to analyze. Region
segmentation is typically used to locate objects and boundaries (lines,
curves, etc.) in images.%

\textbf{Remote sensing.} Sensing something from a distance. The following
processes affect the light that is sensed by a remote sensing system:

% Appendices MUST NOT have sections or subsections
% \section{A Few More Definitions.}

\begin{enumerate}
\item \emph{Illumination}. Light has to illuminate the ground and objects on
the ground before they can reflect any light. In the typical remote sensing
environment, which is outdoors, illumination comes from the sun. We call it
solar illumination.

\item \emph{Atmospheric Absorption and Scattering of Illumination Light}. As
solar illumination travels through the atmosphere, some wavelengths are
absorbed and some are scattered. Scattering is the change in direction of a
light wave that occurs when it strikes a molecule or particle in the atmosphere.

\item \emph{Reflection}. Some of the light that illuminates the ground and
objects on the ground is reflected. The wavelengths that are reflected depend
on the wavelength content of the illumination and on the object's reflectance.
The area surrounding a reflecting object also reflects light, and some of this
light is reflected into the remote sensor.

\item \emph{Atmospheric Absorption and Scattering of Reflected Light}. As
reflected light travels through the atmosphere to the remote sensor, some
wavelengths are absorbed, some are scattered away from the sensor, and some
are scattered into the sensor.
\end{enumerate}

These four effects all change the light that reaches the remote sensor from
the original light source. After the reflected light is captured by the remote
sensor, the light is further affected by how the sensor converts the captured
light into electrical signals. These effects that occur in the sensor are
called sensor effects.

\textbf{Spectral curve.} It is the one-dimensional curve of a spectral reflectance.

\textbf{Spectral libraries.} A spectral library consists of a list of spectral
curves with data of their characteristics corresponding to specific materials
such as mines, plants, etc.

\textbf{Spectral radiance.} It is the measurable reflected light reaching the
sensor. The spectral reflectance of the material is only one factor affecting
it. It is also dependent of the spectra of the input solar energy interactions
of this energy during its downward and upward passages through the atmosphere, etc

\textbf{Shape recognition.} Recognizing the shape of a detected object.

\textbf{Spectral reflectance.} It is the ratio of reflected energy to incident
energy as a function of wavelengths. A certain material has its own spectral
reflectance. The light that is reflected by an object depends on two things:
(1) light that illuminates the object; and (2) the reflectance of the object.
Reflectance is a physical property of the object surface. It is the percentage
of incident EMR of each wavelength that is reflected by the object. Because it
is a physical property, it is not affected by the light that illuminates the object.

\textbf{Spectral reflection.} It is the observed reflected energy, represented
as a function of wavelengths under illumination. It is affected by both
reflectance of the object and the light that illustrates the object. If an
object was illuminated by balanced white light, and if there was no
atmospheric absorption or scatter, and if the sensor was perfect, then the
wavelength composition of reflected light detected by an HSI sensor would
match the reflectance, or spectral signature of the object.

\textbf{Spectral space.} The $n$-dimensional space, where each point is the
spectra of a material or a group of materials.

\textbf{Spectral signature.} A unique characteristic of an object, represented
by some chart of the plot of the object's reflectance as a function of its
wavelength. It can be thought of as an EMR \textquotedblleft
fingerprint\textquotedblright\ of the object.

\textbf{Spectroscopy.} The study of the wavelength composition of
electromagnetic radiation. It is fundamental to how HSI technology works.

\textbf{Signature matching.} Matching reflected light of pixels to spectral
signatures of given objects.

\endinput
%-----------------------------------------------------------------------------
% End of appendix1.tex
%-----------------------------------------------------------------------------

