%-----------------------------------------------------------------------------
% Beginning of abstract.tex
%-----------------------------------------------------------------------------
%
%%%%%%%%%%%%%%%%%%%%%%%%%%%%%%%%%%%%%%%%%%%%%%%%%%%%%%%%%%%%%%%%%%%%%%%%

\chapter*{ABSTRACT}
\addcontentsline{toc}{chapter}{ABSTRACT}

\vspace{-12pt}
\begin{singlespace}
\noindent 
Brown, Connor L. \textit{Ecosystem metabolism of coastal Texas streams across precipitation regimes and land use gradients}. Master of Science (Biology), May, 2022, Sam Houston State University, Huntsville, Texas.
\end{singlespace}

Anthropogenic pressures of land use and climate change have the potential to impact chemical and biological factors that can affect stream ecosystem function. Ecosystem metabolism (i.e., gross primary production [GPP] and ecosystem respiration [ER]), is a metric of stream ecosystem function as it integrates nutrient and carbon cycling. We estimated daily GPP and ER using high temporal frequency oxygen data from nine Texas coastal streams falling along a precipitation and land use gradient. The most arid stream watershed land use consisted of predominantly shrubs and grasses (55\%), whereas the most mesic stream watershed consisted of predominantly agricultural land cover (90\%). These coastal streams did not show strong seasonal variations of GPP or ER, as often found in more temperate regions. GPP ranged from 0.3 g $O_2 m^{-2} d^{-1}$ to 0.9 g $O_2 m^{-2} d^{-1}$, slightly peaking in the middle of the precipitation and land use gradients. ER ranged from -1.0 g $O_2 m^{-2} d^{-1}$ to -4.9 g $O_2 m^{-2} d^{-1}$ with no apparent trend along the precipitation or land use gradient. These results suggest local factors, such as light and nutrients, may be driving ecosystem metabolism, rather than broad scale processes.
\vspace{12pt}
\begin{singlespace} 
\noindent
KEY WORDS: Ecosystem metabolism, Primary production, Ecosystem respiration, Net ecosystem production, Land use.
\end{singlespace}

%\hspace{200pt} Approved:
%\vskip24pt
%\hspace{200pt}\line(1,0){195}\par
%\begin{singlespace}
%\hspace{200pt}Professor Name\par
%\hspace{200pt}Thesis Director\par 
%\end{singlespace}
\endinput
%-----------------------------------------------------------------------------
% End of abstract.tex
%-----------------------------------------------------------------------------
